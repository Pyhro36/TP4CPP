\documentclass[a4paper, 12pts]{article}
\usepackage[top=3.5cm, bottom=3.5cm, left=3cm, right=3cm]{geometry}

\usepackage[T1]{fontenc}
\usepackage[utf8]{inputenc}
\usepackage[francais]{babel}
\usepackage{textcomp}
\usepackage{listings}
%\lstset{language=C++} 

%\usepackage{hyperref} %pour les liens internet

\usepackage{graphicx} %pour les images

\title{TP4 C++ - Cahier Des Charges}
\author{B3129: Pierre-Louis LEFEBVRE et Nicolas SIX}
\date{Vendredi 5 février 2016}

%-----------------------------------------------------------------------------------------


\begin{document}

\begin{titlepage}

\maketitle

\end{titlepage}

%----------------------------------------------Title end

\tableofcontents

\pagebreak

%----------------------------------------------table of contents end

%-------------------------------------
\section{Introduction}

\paragraph{}
Notre logiciel est un éditeur de formes géométriques en ligne de commande (sans interface graphique). L'éditeur permet la gestion des formes géométriques suivantes dans un plan possédant un repère cartésien XY :

\begin{itemize}
	\item Segment de droite
	\item Rectangle aux côtés parallèles aux axes du graphique
	\item Polygone convexe
	\item Réunion et intersection des formes précédentes et d'autres intersections et réunions
\end{itemize}

\paragraph{}
Toutes les formes sont considérées comme « fermées » c'est à dire que les points situés sur les bords de la formes font partie de la forme. Les objets peuvent être vides de tout point (les réunions et intersections notamment).

%-------------------------------------
\section{Commandes}

\subsection{Ajout d'une forme}
 \paragraph{}
Commande
\begin{lstlisting}
	COMMANDE Name PARAM
\end{lstlisting}
\paragraph{}
Réponse
\begin{lstlisting}
	[OK|ERR]
\end{lstlisting}
\paragraph{}
Ajoute une forme de type COMMANDE (voir ci-dessous) à partir des paramètres PARAM (voir ci-dessous) et de nom Name, qui est un mot composé de chiffres et de lettres majuscules ou minuscules uniquement. Les paramètres de type nombre sont toujours des entiers dans la limite de int du C++, sauf indication contraire. La réponse est OK si la commande s'est bien exécutée, ERR sinon. La réponse peut s'accompagner d'un commentaire : une ligne commençant par le caractère \# indiquant dans le cas d'une erreur le type d'erreur (unkown command (lorsque COMMANDE n'est pas une commande du logiciel), name already used (lorsque le nom donné à l'objet à créer est déjà utilisé pour un autre objet), invalid parameters (lorsque les paramètres PARAM est invalide)).

\subsection{Ajout d'un segment}
 \paragraph{}
Commande
\begin{lstlisting}
	S Name X1 Y1 X2 Y2
\end{lstlisting}
\paragraph{}
Ajoute un segment entre les points de coordonnées (X1, Y1), (X2, Y2), X1, Y1, X2, Y2 étant des nombres.

\subsection{Ajout d'un rectangle}
 \paragraph{}
Commande
\begin{lstlisting}
	R Name X1 Y1 X2 Y2
\end{lstlisting}
\paragraph{}
Ajoute un rectangle dont le segment d'extrémités de coordonnées (X1, Y1) et (X2, Y2) est une diagonale.

\subsection{Ajout d'un polygone convexe à n côtés}
 \paragraph{}
Commande
\begin{lstlisting}
	PC Name X1 Y1 X2 Y2 … Xn Yn
\end{lstlisting}
\paragraph{}
Ajoute un polygone de côtés de coordonnées (X1, Y1), (X2, Y2) … (Xn, Yn), avec n ≥ 3 uniquement si ce dernier est convexe, sinon une erreur sera générée avec le commentaire non convex polygon.

\subsection{Ajout d'une réunion de N objets}
 \paragraph{}
Commande
\begin{lstlisting}
	OR Name Name1 Name2 … NameN
\end{lstlisting}
\paragraph{}
Construit un objet Name comme la réunion des N objets Name1, Name2… NameN. N ≥ 0 (si N = 0 alors on crée un objet vide). Si le nom de l'un des N objets ne correspond pas à un objet déjà existant, une erreur de paramètres invalides est renvoyée.

\subsection{Ajout d'une intersection de N objets}
 \paragraph{}
Commande
\begin{lstlisting}
	OI Name Name1 Name2 … NameN
\end{lstlisting}
\paragraph{}
Construit un objet Name comme l'intersection des N objets Name1, Name2… NameN. N ≥ 0 (si N = 0 alors on crée un objet vide). Les erreurs possibles sont les mêmes que pour la réunion.

\subsection{Suppression d'un objet}
 \paragraph{}
Commande
\begin{lstlisting}
	DELETE Name1 Name2 … NameN
\end{lstlisting}
\paragraph{}
Réponse
\begin{lstlisting}
	[OK|ERR]
\end{lstlisting}
\paragraph{}
Supprime les N objets Name1, Name2 … NameN. Si un nom ne correspond pas à un objet existant, aucun objet n'est supprimé et une erreur de paramètres invalides est renvoyée.

\subsection{Déplacement d'un objet}
 \paragraph{}
Commande
\begin{lstlisting}
	MOVE Name dX dY
\end{lstlisting}
\paragraph{}
Réponse
\begin{lstlisting}
	[OK|ERR]
\end{lstlisting}
\paragraph{}
Déplace l'objet Name d'une distance dX (nombre) sur l'axe des X et dY (nombre) sur l'axe des Y

\subsection{Vérification d’appartenance d'un point à l'intérieur d'un objet}
 \paragraph{}
Commande
\begin{lstlisting}
	HIT Name X Y
\end{lstlisting}
\paragraph{}
Réponse
\begin{lstlisting}
	[YES|NO]
\end{lstlisting}
\paragraph{}
Renvoie YES si le point de coordonnées (X, Y) se trouve à l'intérieur (ou au bord) de l'objet Name NO sinon.

\subsection{Énumération des formes géométriques existantes}
 \paragraph{}
Commande
\begin{lstlisting}
	LIST
\end{lstlisting}
\paragraph{}
Réponse
\begin{lstlisting}
	Desc1
	Desc2
	...
	DescN
\end{lstlisting}
\paragraph{}
Affiche les descripteurs des formes géométriques existantes, selon le format suivant :
\paragraph{}
Pour le rectangle, le segment et le polygone, la description est identique à la commande d'ajout correspondante.
\paragraph{Exemple :}Le rectangle de nom Name1 et de côtés opposés (2,3) et (3,8) :
\begin{lstlisting}
	R  Name1 2 3 3 8
\end{lstlisting}
\paragraph{}
Pour l'intersection et la réunion de N objets, la forme est la suivante. Les descripteurs complets de l'ensemble des N objets (y compris ceux d'éventuels autres intersections ou réunion), sauf que leur nom est le nom d'origine immédiatement suivi sans séparateur d'un tiret (-) et du nom de l'intersection ou de la réunion. Suivi de la commande de création de l'intersection ou de la réunion, et enfin de la commande de destruction des N objets.
\paragraph{Exemple :}
La réunion OR1 du rectangle R1, du segment S1 et de l'intersection OI1 du polygone P1 et du segment S2 :
\begin{lstlisting}
	R R1-OR1 1 2 3 8
	S S1-OR1 1 7 45 48
	PC P1-OI1-OR1 14 45 48 12 45 78 89 56 78 10
	S S2-OI1-OR1 0 0 2 2
	OI OI1-OR1 P1-OI1-OR1 S2-OI1-OR1  
	DELETE  P1-OI1-OR1
	DELETE  S2-OI1-OR1
	OR OR1 R1-OR1 S1-OR1 OI1-OR1
	DELETE  R1-OR1
	DELELTE  S1-OR1
	DELETE  OI1-OR1
\end{lstlisting}
\paragraph{}
La persistance d'un ensemble d'objets construit de cette manière ; le fichier
de sauvegarde est un fichier de format texte.

\end{document}












