\part{Conception}

\section{Détection des d’appartenance d'un point à une forme}

\subsection{Appliquer à un ensemble}
\paragraph{}
Chaque ensemble, intersections et réunions, contient chacun des éléments qui le compose. Pour effectuer la détection un ensemble se contente donc de vérifier si le point donné appartient ou non aux éléments qui le compose. Cette opération peut sembler coûteuse mais permet de ne pas avoir les problèmes d’imprécision due aux arrondis qui seraient apparus lors du calcul d'une forme qui résumerait cet ensemble. De plus il est facile de ne pas étudier chaque sous-forme dans les nombreux cas que représentent les réponses négatives.
\paragraph{}
Il apparaît donc qu'une détection efficace des hit sur les formes élémentaire est primordiale.
\paragraph{coût: }
\(O(n)\) avec n le nombre de figure contenu dans l'ensemble.

\subsection{Appliquer à un segment}
\paragraph{}
La détection des hit sur un segment se base sur l’équation de la droite correspondant au segment avec tout d'abord vérification de l’appartenance au rectangle dont la diagonale est le segment considéré. Cette méthode est relativement complexe, car elle nécessite en plus de quelques additions une multiplication et surtout une division, toute fois ces calculs restent rapides et ne posent pas de problèmes de performance.
\paragraph{coût: }
\(O(1)\)

\subsection{Appliquer à un rectangle}
\paragraph{}
La détection des hit sur un rectangle et ici triviale vu qu'elle se compose de quelques comparaisons par rapport aux valeurs max et min de la figure suivant les deux axes une opération très rapide.
\paragraph{coût: }
\(O(1)\)

\subsection{Appliquer à un Polygone convexe}
\paragraph{}
La détection des hit sur un polygone convexe détermine si un point appartient à la figure en cherchant un segment se situant au-dessus du point et un en dessous. La grande majorité des cas sont traités en quelques conditions. Les cas les plus compliqués quant à eux utilisent le même système que pour les segments en utilisant l'équation du côté considéré. Cette méthode traite donc les côtés un par un juste à en trouver un au-dessus et un en dessous ou bien jusqu’à trouver un point appartenant à un côté. Bien que relativement lourds sur les polygones ayant de nombreux cotés, cette technique reste capable de traiter tous les côtés sauf dans le pire des cas deux d'entre eux juste en faisant quelques comparaisons de valeur ce qui lui permet de rester très efficace.
\paragraph{coût: }
\(O(n)\) avec n le nombre de points du polygone.

\section{Détection de la convexité d'un polygone}
\paragraph{}
Afin de vérifier qu'un polygone est bien convexe, nous vérifions que le sinus de l'angle entre deux points formant un côté du polygone et un autre point du polygone reste toujours de même signe opération que nous répétons pour toutes les combinaisons possibles dans le polygone. Cette opération est assez lourde en calcul mais nous assure que le polygone et bien convexe et nous permet notamment de détecter les polygones croisés et repliées sur eux-mêmes.
\paragraph{coût: }
\(O(n^2)\) avec n le nombre de points du polygone.




